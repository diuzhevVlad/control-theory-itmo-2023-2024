\tableofcontents
\section*{Предисловие}
При выполнении данной лабораторной работы было решено использовать 
\href{https://python-control.readthedocs.io/en/0.9.4/}{Python Control Systems Library}.
Данный инструмент является альтернативой Matlab, адаптированной для использования на 
языке Python и предоставляет широкий функционал для анализа и моделирования систем,
а также синтеза регуляторов для управления.

Полный листинг моделирования систем представлен в \href{https://github.com/diuzhevVlad/control-theory-itmo-fall-2023/blob/main/Lab7/Lab7.ipynb}{jupyter notebook} на GitHub.

\pagebreak


\section{Полностью управляемая система}
Рассмотрим система, заданную матрицами $A$ и $B$:
\begin{equation*}
    A = \begin{bmatrix}
        7 & -7 & 8 \\
        6 & -5 & 6 \\
        -6 & 4 & -7
    \end{bmatrix},
    B = \begin{bmatrix}
        -4 \\ -2 \\ 4
    \end{bmatrix}
\end{equation*}

\subsection{Матрица управляемости}
Запишем матрицй управляемости системы:
\begin{equation*}
    U = \begin{bmatrix}
        B & AB & A^2B 
    \end{bmatrix}
     = 
    \begin{bmatrix}
        -4 & 18 & -40 \\
        -2 & 10 & -14 \\
        4 & -12 & 16
    \end{bmatrix}
\end{equation*}
Заметим, что ранг данной матрицы равен 3, следовательно, система - полностью управляема.

\subsection{Жорданова форма}
Представим систему в Жордановом базисе:
\begin{equation}
    \begin{cases}
        \dot{\hat{x}} = P^{-1}AP\hat{x} + P^{-1}Bu  \\
        y = CP\hat{x}
    \end{cases}
\end{equation}
где $P$ - матрица обобщенных векторов. ЖНФ матрицы $A$:
\begin{equation*}
    A = PJP^{-1} = \begin{bmatrix}
        -1 & -\frac{3}{2} + \frac{i}{2} & -\frac{3}{2}-\frac{i}{2} \\
        0 & -1 & -1 \\
        1 & 1 & 1
    \end{bmatrix}
    \begin{bmatrix}
        -1 & 0 & 0 \\
        0 & -2-3i & 0 \\
        0 & 0 & -2+3i
    \end{bmatrix}
    \begin{bmatrix}
        -1 & -\frac{3}{2} + \frac{i}{2} & -\frac{3}{2}-\frac{i}{2} \\
        0 & -1 & -1 \\
        1 & 1 & 1
    \end{bmatrix}^{-1}
\end{equation*}
Матрица входных воздействий в Жордановом базисе:
\begin{equation*}
    P^{-1}B = \begin{bmatrix}
        2 \\ 1-i \\ 1+i
    \end{bmatrix}
\end{equation*}
Все жордановы клетки матрицы $J$ соответсвуют разным собственным числам и элементы матрицы входных воздействий
соответствующие концам клеток не равны нулю. Следовательно: все собственные чила - управляемы.

Также, заметим:
\begin{equation*}
    \begin{cases}
        rank(\begin{bmatrix}
            A - (-1)\cdot I &  B
        \end{bmatrix} ) = 3 \\
        rank(\begin{bmatrix}
            A - (-2-3i)\cdot I &  B
        \end{bmatrix} ) = 3 \\
        rank(\begin{bmatrix}
            A - (-2+3i)\cdot I &  B
        \end{bmatrix} ) = 3
    \end{cases}
\end{equation*}
что подтверждает управляемость всех собственных чисел.

\subsection{Управляемое подпространство}
Т.к. система - полностью управляема, управляемое подпространство совпадает с $\mathcal{R}^3$ и любой вектор
принадлежит ему, в том числе $x_1$:
\begin{equation*}
    x_1 = \begin{bmatrix}
        5 \\
        3 \\
        -3
    \end{bmatrix}
\end{equation*}

\subsection{Грамиан управляемости}
Расчитаем грамиан управляемости системы:
\begin{equation}
    P(t_1) = \int_{0}^{t_1}e^{At}BB^Te^{A^Tt}dt
\end{equation}
\begin{equation*}
    P(3) = \begin{bmatrix}
        1.777 & 0.654 & -1.982 \\
        0.654 & 0.538 & -0.538 \\
        -1.982 & -0.539 & 2.534
        \end{bmatrix}
\end{equation*}
Собственные числа грамиана:
\begin{eqnarray*}
    \lambda_1 = 0.055, \lambda_2 = 0.439, \lambda_3 = 4.355
\end{eqnarray*}

\subsection{Расчет управления}
Расчитаем управление, необходимое для перехода из нулевого состояния в состояние $x_1$ за $t_1$ секунд:
\begin{equation}
    u(t) = B^Te^{A^T(t_1-t)}(P(t_1))^{-1}x_1
\end{equation}
Искомое управление:
\begin{equation*}
    u(t) = B^Te^{A^T(3-t)}(P(3))^{-1}x_1 = -9.196e^{t-3} - 32.765e^{2t-6}sin(3t-9)-5.244e^{2t-6}cos(3t-9)
\end{equation*}

\subsection{Моделирование}
